% \section{}

% \begin{problem}
%     $G$是一个群,$a$是$G$的元素,$Z(a)=\{gag^{-1}=a|g\in G\}$是$G$的子群. 现将$G$按$Z(a)$作陪集分解$G=Z(a)\oplus x_1Z(a)\oplus\cdots$,证明
%     \begin{enumerate}
%         \item 若属于同一个陪集,则$g_1ag_1^{-1}=g_2ag_2^{-1}$,反之$g_1ag_1^{-1}\neq g_2ag_2^{-1}$.
%         \item $a$的共轭类包含的元素数目等于$G$的阶数除以$Z(a)$的阶数.
%     \end{enumerate}
% \end{problem}
% \begin{solution}

% \end{solution}

% \begin{problem}
%     $G$是一个群,$a$是$G$的元素.
%     \begin{enumerate}
%         \item 证明$G$必含有如下循环子群$H=\{e,a,a^2,\ldots,a^k,\ldots\}$.
%         \item 若$G$的阶数是素数,则$G$必是循环群.
%     \end{enumerate}
% \end{problem}

% \section{}

% \begin{problem}
%     群元$R$对应的算符$P_R$满足 $P_Rf\left(r\right)=f(R^{-1}r)$,$f(r)$是表示空间的函数. 现已知五个球谐函数$Y_{2m},\ -2\leq m\leq 2$是转动群表示的一组基, 求绕x轴逆时针转$90^\circ$的群元在这组基下的表示矩阵.
% \end{problem}
% \begin{solution}
    
% \end{solution}

% \begin{problem}
%     证明:与Hamiltonian对易的所有幺正算符构成群.
% \end{problem}
% \begin{solution}
%     \begin{proof}
        
%     \end{proof}
% \end{solution}

% \section{}

% \begin{problem}
%     证明$(A\otimes B)(C\otimes D)=AC\otimes BD$,其中$\otimes$表示直积.
% \end{problem}
% \begin{solution}
% \begin{proof}
    
% \end{proof}
% \end{solution}

% \begin{problem}
%     一维周期性晶格,晶格常数为$a$,其中的电子Hamiltonian为
%     \[\mathcal H=-\frac{\hbar^2}{2m}\nabla^2+V(r)\]

%     势能$V(r)$具有晶格周期性$V(r+na)=V(r)$,$n$为任意整数. 定义平移算符$T_nf(r)=f(r+na)$,$f(r)$是任意函数. 证明
%     \begin{enumerate}
%         \item $T_n$与$\mathcal H$对易.
%         \item 所有$T_n$构成一个Abel群.
%         \item 上述的群的不可约表示都是$1$维的.
%     \end{enumerate}
% \end{problem}
% \begin{solution}
% \begin{proof}
    
% \end{proof}
% \end{solution}

% \section{}

% \begin{problem}
%     $O$群是立方体的对称群,它共有$24$个元素,$5$个共轭类,其特征标表如下

%     \begin{table}[!ht]
%         \centering
%         \begin{tabular*}{.6\linewidth}{@{\extracolsep{\fill}}c|ccccc}
%             \toprule
%             $O$ & $E$ & $3C_2^4$ & $8C_3'$ & $6C_4$ & $6C_2''$\\
%             \bottomrule
%             $A$ & $1$ & $1$ & $1$ & $1$ & $1$\\
%             $B$ & $1$ & $1$ & $1$ & $-1$ & $-1$\\
%             $E$ & $2$ & $2$ & $-1$ & $0$ & $0$\\
%             $T_1$ & $3$ & $-1$ & $0$ & $1$ & $-2$\\
%             $T_2$ & $3$ & $-1$ & $0$ & $-1$ & $1$\\
%             \bottomrule
%         \end{tabular*}
%     \end{table}
%     \begin{enumerate}
%         \item 分别是几维表示.
%         \item 可以约化成哪些不可约表示的直和.
%     \end{enumerate}
% \end{problem}
% \begin{solution}
    
% \end{solution}

% \section{}
% \begin{problem}
%     数学补充.
%     \[\mathbf A=\begin{pmatrix}
%         A_1\\A_2\\A_3
%     \end{pmatrix},\ \mathbf B=\begin{pmatrix}
%         B_1\\B_2\\B_3
%     \end{pmatrix}\]

%     则并矢$\mathbf A\mathbf B=\begin{pmatrix}
%         A_1\\A_2\\A_3
%     \end{pmatrix}\begin{pmatrix}
%         B_1 & B_2 & B_3
%     \end{pmatrix}=\begin{pmatrix}
%         A_1B_1 & A_1B_2 & A_1B_3\\
%         A_2B_1 & A_2B_2 & A_2B_3\\
%         A_3B_1 & A_3B_2 & A_3B_3
%     \end{pmatrix}$.

%     $C_1$是绕$x$轴转$\pi$角的操作,$n$是$xoy$平面内与$x$轴成$\alpha$角的转轴,$C_2$是绕$n$轴转$\pi$角的操作. 证明$C_1$与$C_2$的乘积是绕$z$轴转$2\alpha$角的操作
% \end{problem}
% \begin{solution}
% \begin{proof}
    
% \end{proof}
% \end{solution}

% \section{}

% \begin{problem}
%     $S_3=\begin{pmatrix}
%         1&0&0\\0&0&0\\0&0&-1
%     \end{pmatrix}$. 计算$\exp(-\i\alpha S_3)$
% \end{problem}
% \begin{solution}
    
% \end{solution}

% \section{}

% \begin{problem}
%     李群$G$有$n$个生成元$T_1,\ldots,T_n$,群元可以写为$g=e^{-\i\omega_kT_k}$(默认对重复指标求和),若所有群元$g$均与某个Hamiltonian $H$对易,证明:$T_1\ldots T_n$与$H$对易.
% \end{problem}

% \section{}

% \begin{problem}
%     正交矩阵是满足$R^TR=I$的矩阵($R^T$是$R$的转置),证明
%     \begin{enumerate}
%         \item 所有的N维正交矩阵构成群,称为$O(N)$群.
%         \item $O(N)$群的自由参数数目为$N(N-1)/2$.
%         \item $O(N)$群的无穷小生成元是反对称矩阵.
%     \end{enumerate}
% \end{problem}
% \begin{solution}
    
% \end{solution}

% \begin{problem}
%     $u=\exp\ab(-\frac{\i\sigma_x\alpha}{2})$, 求$u\sigma_xu^{-1},u\sigma_yu^{-1}$,$u\sigma_zu^{-1}$
% \end{problem}
% \begin{solution}
    
% \end{solution}

% \section{}

% \begin{problem}
%     证明$u(\hat{\bm n}_1,\pi)u(\hat{\bm n}_2,\pi)=-1(\hat{\bm n_1\cdot\hat{\bm n_2}})-\i\bm\sigma\cdot(\hat{\bm n_1}\times\hat{\bm n_2})$. 其中$u(n,\pi)=\exp\ab(-\frac{\i(\bm\sigma\cdot n)\pi}{2})$.
% \end{problem}
% \begin{solution}
% \begin{proof}
    
% \end{proof}
% \end{solution}